\documentclass[12pt,xgraphicx=dvips,xcolor=dvips]{beamer}
% pdfの栞の字化けを防ぐ
% \AtBeginDvi{\special{pdf:tounicode EUC-UCS2}}
% テーマ
\usepackage{graphicx}
\usetheme{Warsaw}
% navi. symbolsは目立たないが,dvipdfmxを使うと機能しないので非表示に
\setbeamertemplate{navigation symbols}{}
\setbeamertemplate{footline}[frame number]
\usepackage{amsmath}
\usepackage{amssymb}
\usepackage{pstricks}
% フォントはお好みで
\usepackage{txfonts}
\mathversion{bold}
\renewcommand{\familydefault}{\sfdefault}
\renewcommand{\kanjifamilydefault}{\gtdefault}
\setbeamerfont{title}{size=\large,series=\bfseries}
\setbeamerfont{frametitle}{size=\large,series=\bfseries}
\setbeamertemplate{frametitle}[default][center]
\usefonttheme{professionalfonts}
\usecolortheme{beaver}
%

%%
\makeatletter
\renewcommand\verbatim@font{\footnotesize\ttfamily}
\makeatother
\setbeamerfont{footline}{size=\small,series=\bfseries}
\setbeamercolor{footline}{fg=black,bg=black}
\setbeamersize{text margin left=1zw, text margin right=1zw}
\setbeamertemplate{items}[default]
\setbeamercolor{block title example}{bg=darkred!60!black,fg=white}
\setbeamercolor{block body example}{bg=gray!10!white}
%%

\title{Boost.勉強会 \#7\\clangで入門 解析戦略ー}
\author{藤田 典久: {\tt @fjnli, id:fjnl}}
\date{2011/12/03}

\begin{document}

\frame{\titlepage}

\begin{frame}
  \frametitle{Agenda}

  \setbeamertemplate{items}{}
  \begin{itemize}
    \setlength{\itemsep}{0.5zh}
    \item {\tt https://github.com/fjnl/\\boost-study-7-Tokyo/raw/master/sheet.pdf} \\
    \item {\tt http://bit.ly/uCnZxE}
  \end{itemize}
  \setbeamertemplate{items}[default]

  \vspace{1zh}
  \hrule
  \vspace{.5zh}

  \begin{enumerate}
    \setlength{\itemsep}{1.5zh}
    \item About LLVM and clang
    \item About libclang
    \item 定義元を調べるツールを作ってみる
    \item 補足, $+\alpha$
  \end{enumerate}
\end{frame}

\begin{frame}
  \frametitle{自己紹介}

  \setbeamertemplate{items}{}
  \begin{itemize}
    \setlength{\itemsep}{0.5zh}
    \item {\tt https://github.com/fjnl/\\boost-study-7-Tokyo/raw/master/sheet.pdf} \\
    \item {\tt http://bit.ly/uCnZxE}
  \end{itemize}
  \setbeamertemplate{items}[default]

  \vspace{1zh}
  \hrule
  \vspace{.5zh}

  \begin{itemize}
    \setlength{\itemsep}{1.5zh}
    \item 某大学で院生やってます {\scriptsize 進路どうしよ}
    \item 研究はHPCとかGPGPUとか
    \item Twitter $\rightarrow$ {\tt @fjnli}
    \item はてな $\rightarrow$ {\tt id:fjnl}
      \begin{itemize}
        \item C++11 Advent Calender 6日目
        \item Boost Advent Calender 7日目
      \end{itemize}
  \end{itemize}

  \rput[c](9,3){\fbox{\includegraphics{icon.eps}}}
\end{frame}

\begin{frame}
  \frametitle{LLVM}

  \begin{itemize}
    \setlength{\itemsep}{1.5zh}
    \item \alert{L}ow \alert{L}evel \alert{V}irtual \alert{M}achine
    \item {\tt http://llvm.org/}
    \item 特定の言語,アーキテクチャに依存しない中間言語
    \item 言語
      \begin{itemize}
        \item clang, llvm-gcc (C, C++, Obj-C)
        \item ghc (Haskell)
        \item MacRuby (Ruby) など
      \end{itemize}
    \item アーキテクチャ
      \begin{itemize}
        \item x86, x86\_64
        \item PowerPC
        \item ARM など
      \end{itemize}
  \end{itemize}
\end{frame}

\begin{frame}
  \frametitle{clang}

  \begin{itemize}
    \setlength{\itemsep}{1.5zh}
    \item {\tt http://clang.llvm.org/}
    \item C + Obj-C + C++ コンパイラ期待の\alert{新星}
    \item LLVMをバックエンドとしている
    \item たぶん ``クラン'' と読む (×クラング)
    \item BSD License
    \item Apple (Mac OS/iOS)とかFreeBSDとか
  \end{itemize}
\end{frame}

\begin{frame}[containsverbatim]
  \frametitle{}

  \begin{exampleblock}{C++}
\begin{semiverbatim}
int fib(int n) \{
  if (n == 0) return 0;
  if (n == 1) return 1;
  return fib(n - 1) + fib(n - 2);
\}
\end{semiverbatim}
  \end{exampleblock}

  \begin{exampleblock}{LLVM Byte Code}
\begin{semiverbatim}
define i32 @fib(i32 %n) nounwind ssp \{
  %1 = alloca i32, align 4
  %2 = alloca i32, align 4
  store i32 %n, i32* %2, align 4
  %3 = load i32* %2, align 4
  %4 = icmp eq i32 %3, 0
  br i1 %4, label %5, label %6
  (略)
\}
\end{semiverbatim}
  \end{exampleblock}

\vspace{-1zh}
\begin{pspicture}(0,0)
  \psline[linecolor=red,linewidth=2pt]{->}(7,6)(8,6)(8,3)(7,3)
  \rput[l](8.1,5.5){\color{red} clang}

  \psline[linewidth=2pt]{->}(7,2)(8,2)(8,-.5)
  \rput[l](8.1,1.5){LLVM}
\end{pspicture}

\end{frame}

\begin{frame}
  \frametitle{Why?}

  \begin{itemize}
    \setlength{\itemsep}{1.5zh}
    \item gccのライセンスが GPLv2 $\rightarrow$ GPLv3 になってしまった
    \item 最後のGPLv2のgccが4.2.1
    \item 一部のシステムではgccの更新をやめてしまった
      \begin{itemize}
        \item MacOS
        \item *BSD
      \end{itemize}
    \item 注: バンドルされてないだけで,自分で入れれば最新gccは使える
  \end{itemize}
\end{frame}

\begin{frame}
  \frametitle{Why?}

  \begin{itemize}
    \setlength{\itemsep}{1.5zh}
    \item 長らく代替コンパイラが不在だった
    \item Apple $\rightarrow$ XCode4からデフォルト
    \item FreeBSD $\rightarrow$ 徐々にclangのサポートを拡大中
      \begin{itemize}
        \item FreeBSD 9にマージされた
        \item 現在 FreeBSD 9.0-RELEASE RC2
        \item デフォルトはまだgcc
      \end{itemize}
  \end{itemize}
\end{frame}

\begin{frame}
  \begin{center}
    さて,ここからはclangのお話…ではありません
  \end{center}

  \pause

  \begin{center}
    libclangがメインです
  \end{center}

  \pause

  \begin{center}
    あとBoostは出てきません
  \end{center}
\end{frame}

\begin{frame}
  \frametitle{libclang}

  \begin{itemize}
    \setlength{\itemsep}{1.5zh}
    \item clangのバックエンドがライブラリ化されたもの
      \begin{itemize}
        \item 注: 僕が勝手に付けた名前です
      \end{itemize}
    \item 何ができるの? $\rightarrow$ C++コードの解析
      \begin{itemize}
        \item 抽象構文木 (Abstract Syntax Tree; AST) の取得
        \item clang static analyzer
        \item {\tt http://clang-analyzer.llvm.org/}
      \end{itemize}
  \end{itemize}
\end{frame}

\begin{frame}
  \frametitle{libclang}

  \begin{description}[Cons.]
    \setlength{\itemsep}{1.5zh}
    \item[Pros.]
      \begin{itemize}
        \setlength{\itemsep}{0.75zh}
        \item 開発が非常に活発
        \item コンパイラのコードをそのまま流用できる
        \item C++11に対応しつつある
      \end{itemize}
    \item[Cons.]
      \begin{itemize}
        \setlength{\itemsep}{0.75zh}
        \item Not user friendly
        \item 解析のみを欲するにはやや巨大
      \end{itemize}
  \end{description}
\end{frame}

\begin{frame}
  \frametitle{実験環境}

  \begin{columns}

    \begin{column}[t]{.40\textwidth}
      \begin{itemize}
        \setlength{\itemsep}{1.5zh}
        \item LLVM 2.9
        \item clang 2.9
        \item gcc 4.6.1
        \item C++ 11
      \end{itemize}
    \end{column}

    \pause

    \begin{column}[t]{.40\textwidth}
      \begin{itemize}
        \setlength{\itemsep}{1.5zh}
        \item 11月30日にLLVM 3.0および
        \item clang 3.0が出てるはず
        \item Windows環境で開発しないのでわかりません.ごめんなさい.
      \end{itemize}
    \end{column}
  \end{columns}

%  \psgrid[gridcolor=gray,griddots=5,subgriddots=4](0,0)(12,9)
%  \pscurve{-}(6.6,3.6)(6.4,2.6)(6.6,1.6)
\end{frame}

\begin{frame}[containsverbatim]
  \frametitle{Target}

  \begin{semiverbatim}
    $ ./where in.cpp f
      in.cpp:2:5

    $ ./where in.cpp g
      in.cpp:6:6
  \end{semiverbatim}

  \begin{exampleblock}{in.cpp}
    \begin{semiverbatim}
1: #include <iostream>
2: int f(int x, int y) \{
3:   return x + y;
4: \}
5:
6: void g() \{
7:  std::cout << ``g()'' << std::endl;
8: \}
    \end{semiverbatim}
  \end{exampleblock}
\end{frame}

\begin{frame}
  \frametitle{コンパイル方法}

  \setbeamertemplate{items}{}

  \begin{columns}
    \begin{column}{.55\textwidth}
      \begin{itemize}
        \setlength{\itemsep}{1.5zh}
      \item {\tt gcc -o where where.cpp} $\backslash$
        %    \item {\tt -fno-rtti} $\backslash$
        %    \item {\tt -fno-exceptions} $\backslash$
      \item {\tt -D\_\_STDC\_CONSTANT\_MACROS} $\backslash$
      \item {\tt -D\_\_STDC\_LIMIT\_MACROS} $\backslash$
      \item libLLVM*.a libclang*.a
      \end{itemize}
    \end{column}

    \begin{column}[t]{.45\textwidth}
    \end{column}
  \end{columns}

  \setbeamertemplate{items}{default}
\end{frame}

\begin{frame}
  \frametitle{コンパイル方法}

  \setbeamertemplate{items}{}

  \begin{columns}
    \begin{column}{.55\textwidth}
      \begin{itemize}
        \setlength{\itemsep}{1.5zh}
      \item {\tt gcc -o where where.cpp} $\backslash$
        %    \item {\tt -fno-rtti} $\backslash$
        %    \item {\tt -fno-exceptions} $\backslash$
      \item {\tt -D\_\_STDC\_CONSTANT\_MACROS} $\backslash$
      \item {\tt -D\_\_STDC\_LIMIT\_MACROS} $\backslash$
      \item libLLVM*.a libclang*.a
      \end{itemize}
    \end{column}

    \begin{column}[t]{.45\textwidth}
      \vspace{-2zh}
      \begin{exampleblock}{\scriptsize \_\_STDC\_CONSTANT\_MACROS}
        {\tt INT8\_C}, {\tt INT16\_C}, {\tt INT32\_C}, {\tt INT64\_C}, ...
      \end{exampleblock}
    \end{column}
  \end{columns}

  \psline[linewidth=1pt,linecolor=darkred!60!black]{-}(0.6,2.2)(6.6,2.2)

  \setbeamertemplate{items}{default}
\end{frame}

\begin{frame}
  \frametitle{コンパイル方法}

  \setbeamertemplate{items}{}

  \begin{columns}
    \begin{column}{.55\textwidth}
      \begin{itemize}
        \setlength{\itemsep}{1.5zh}
      \item {\tt gcc -o where where.cpp} $\backslash$
        %    \item {\tt -fno-rtti} $\backslash$
        %    \item {\tt -fno-exceptions} $\backslash$
      \item {\tt -D\_\_STDC\_CONSTANT\_MACROS} $\backslash$
      \item {\tt -D\_\_STDC\_LIMIT\_MACROS} $\backslash$
      \item libLLVM*.a libclang*.a
      \end{itemize}
    \end{column}

    \begin{column}[t]{.45\textwidth}
      \vspace{1.0zh}
      \begin{exampleblock}{\scriptsize \_\_STDC\_LIMIT\_MACROS}
        {\tt INT8\_MAX}, {\tt INT8\_MIN}, {\tt INT16\_MAX}, {\tt INT16\_MIN} ...
      \end{exampleblock}
    \end{column}
  \end{columns}

  \psline[linewidth=1pt,linecolor=darkred!60!black]{-}(0.6,1.5)(6.6,1.5)

  \setbeamertemplate{items}{default}
\end{frame}

\begin{frame}
  \frametitle{コンパイル方法}

  \setbeamertemplate{items}{}

  \begin{columns}
    \begin{column}{.55\textwidth}
      \begin{itemize}
        \setlength{\itemsep}{1.5zh}
      \item {\tt gcc -o where where.cpp} $\backslash$
        %    \item {\tt -fno-rtti} $\backslash$
        %    \item {\tt -fno-exceptions} $\backslash$
      \item {\tt -D\_\_STDC\_CONSTANT\_MACROS} $\backslash$
      \item {\tt -D\_\_STDC\_LIMIT\_MACROS} $\backslash$
      \item libLLVM*.a libclang*.a
      \end{itemize}
    \end{column}

    \begin{column}[t]{.45\textwidth}
      \vspace{-2zh}
      \begin{exampleblock}{\scriptsize lib*.a}
        libLLVMAnalysis.a, libLLVMArchive.a libLLVMAsmParser.a, ... \\
        libclang.a libclangAST.a libclangAnalysis.a libclangBasic.a, ...
      \end{exampleblock}
    \end{column}
  \end{columns}

  \psline[linewidth=1pt,linecolor=darkred!60!black]{-}(0.6,1.2)(6.1,1.2)(6.1,3.5)(6.6,3.5)

  \setbeamertemplate{items}{default}
\end{frame}

\begin{frame}[containsverbatim]
  \frametitle{where.cpp:\#include}
  \begin{semiverbatim}
#include <clang/AST/ASTContext.h>
#include <clang/AST/ASTConsumer.h>
#include <clang/AST/Decl.h>
#include <clang/AST/DeclGroup.h>
#include <clang/Basic/TargetInfo.h>
#include <clang/Frontend/CompilerInstance.h>
#include <clang/Frontend/CompilerInvocation.h>
#include <clang/Parse/ParseAST.h>
#include <clang/Lex/Preprocessor.h>
    \end{semiverbatim}
\end{frame}

\begin{frame}[containsverbatim]
  \frametitle{where.cpp:main}
  \begin{semiverbatim}
int main(int argc, char** argv) \{
  clang::CompilerInstance compiler;
  compiler.createDiagnostics(argc - 1, argv);
  auto& diag = compiler.getDiagnostics();
  auto& invocation = compiler.getInvocation();

  clang::CompilerInvocation::CreateFromArgs(
    invocation, argv + 1, argv + argc - 1, diag);
  compiler.setTarget(clang::TargetInfo::CreateTargetInfo(
    diag, compiler.getTargetOpts()));

  compiler.createFileManager();
  compiler.createSourceManager(compiler.getFileManager());
  compiler.createPreprocessor();
  compiler.createASTContext();
  compiler.setASTConsumer(new consumer(argv[argc - 1]));
  compiler.createSema(false, nullptr);
  (続)
  \end{semiverbatim}
\end{frame}

\begin{frame}[containsverbatim]
  \frametitle{where.cpp:main}
  \begin{semiverbatim}
  (続)
  auto& pp = compiler.getPreprocessor();
  pp.getBuiltinInfo().InitializeBuiltins(
      pp.getIdentifierTable(), pp.getLangOptions());

  auto& inputs = compiler.getFrontendOpts().Inputs;
  if (inputs.size() > 0) \{
    compiler.InitializeSourceManager(inputs[0].second);
    clang::ParseAST(
      pp,
      &compiler.getASTConsumer(),
      compiler.getASTContext()
    );
  \}
  return 0;
\}
  \end{semiverbatim}
\end{frame}

\begin{frame}[containsverbatim]
  \frametitle{clang::CompilerInstance}
  \begin{semiverbatim}
int main(int argc, char** argv) \{
  clang::CompilerInstance compiler;
\color{gray}{  compiler.createDiagnostics(argc - 1, argv);
  auto& diag = compiler.getDiagnostics();
  auto& invocation = compiler.getInvocation();

  clang::CompilerInvocation::CreateFromArgs(
    invocation, argv + 1, argv + argc - 1, diag);
  compiler.setTarget(clang::TargetInfo::CreateTargetInfo(
    diag, compiler.getTargetOpts()));

  compiler.createFileManager();
  compiler.createSourceManager(compiler.getFileManager());
  compiler.createPreprocessor();
  compiler.createASTContext();
  compiler.setASTConsumer(new consumer(argv[argc - 1]));
  compiler.createSema(false, nullptr);}
  (続)
  \end{semiverbatim}


%  \psgrid[gridcolor=gray,griddots=5,subgriddots=4](0,0)(12,9)

  \psframe[linewidth=0pt,fillcolor=gray!10!white,fillstyle=solid,framearc=.2](4.8,7.1)(11.5,6.3)
%%  \psline[linewidth=2pt,linecolor=darkred!60!black]{-}(0.3,7.7)(6.5,7.7)
  \psline[linewidth=2pt,linecolor=darkred!60!black]{*->}(4.8,7.1)(4,7.1)(3.7,7.6)
  \rput[lb](5.0,6.5){clangコンパイラ本体みたいなもの}
\end{frame}

\begin{frame}[containsverbatim]
  \frametitle{CompilerInvocation::CreateFromArgs}
  \begin{semiverbatim}
\color{gray}{int main(int argc, char** argv) \{
  clang::CompilerInstance compiler;
  compiler.createDiagnostics(argc - 1, argv);
  auto& diag = compiler.getDiagnostics();
  auto& invocation = compiler.getInvocation();}

\color{black}{  clang::CompilerInvocation::CreateFromArgs(
    invocation, argv + 1, argv + argc - 1, diag);}
\color{gray}{  compiler.setTarget(clang::TargetInfo::CreateTargetInfo(
    diag, compiler.getTargetOpts()));

  compiler.createFileManager();
  compiler.createSourceManager(compiler.getFileManager());
  compiler.createPreprocessor();
  compiler.createASTContext();
  compiler.setASTConsumer(new consumer(argv[argc - 1]));
  compiler.createSema(false, nullptr);}
  (続)
  \end{semiverbatim}


%  \psgrid[gridcolor=gray,griddots=5,subgriddots=4](0,0)(12,9)

  \psframe[linewidth=0pt,fillcolor=gray!10!white,fillstyle=solid,framearc=.2](3,5.1)(9,2)
%  \psline[linewidth=2pt,linecolor=darkred!60!black]{-}(0.3,7.7)(6.5,7.7)
  \psline[linewidth=2pt,linecolor=darkred!60!black]{*->}(4.8,5.1)(5.0,5.4)(5.4,5.6)
  \rput[lb](3.1,4.6){clang本体の引数解析に}
  \rput[lb](3.1,4.2){丸投げする.}
  \rput[lb](3.1,3.8){楽だが不便でもある.}
  \rput[lb](3.1,3.0){({\tt -D -I -U}といったフラグ}
  \rput[lb](3.1,2.6){を認識させる)}
\end{frame}

\begin{frame}[containsverbatim]
  \frametitle{CompilerInvocation::CreateFromArgs}

  \begin{semiverbatim}
clang::CompilerInvocation::CreateFromArgs(
  invocation, argv + 1, argv + argc - 1, diag);
  \end{semiverbatim}

  ※ {\tt argc == 1}の時まずいが今回は気にしない方向で.
  \vspace{1zh}

  \begin{itemize}
    \setlength{\itemsep}{1.5zh}
    \item 馴染みあるオプションを丸投げできる
      \setbeamertemplate{items}[]
      \begin{itemize}
        \item {\tt -I\$HOME/boost}
        \item {\tt -DNDEBUG}
        \item {\tt -UFOOBAR} ...
      \end{itemize}
      \setbeamertemplate{items}[default]
    \item アプリ独自のフラグを渡すのが手間 (see: argv + argc - 1)
  \end{itemize}
\end{frame}

\begin{frame}[containsverbatim]
  \frametitle{CompilerInvocation::CreateFromArgs}
  \begin{semiverbatim}
\color{gray}int main(int argc, char** argv) \{
  clang::CompilerInstance compiler;
\color{black}  compiler.createDiagnostics(argc - 1, argv);
  auto& diag = compiler.getDiagnostics();
  auto& invocation = compiler.getInvocation();

\color{gray}  clang::CompilerInvocation::CreateFromArgs(
    invocation, argv + 1, argv + argc - 1, diag);
\color{black}  compiler.setTarget(clang::TargetInfo::CreateTargetInfo(
    diag, compiler.getTargetOpts()));

  compiler.createFileManager();
  compiler.createSourceManager(compiler.getFileManager());
  compiler.createPreprocessor();
  compiler.createASTContext();
  compiler.setASTConsumer(new consumer(argv[argc - 1]));
  compiler.createSema(false, nullptr);
  (続)
  \end{semiverbatim}
\end{frame}

\begin{frame}[containsverbatim]
  \frametitle{InitializeBuiltins}
  \begin{semiverbatim}
  (続)
  auto& pp = compiler.getPreprocessor();
  pp.getBuiltinInfo().InitializeBuiltins(
      pp.getIdentifierTable(), pp.getLangOptions());

  \color{gray}auto& inputs = compiler.getFrontendOpts().Inputs;
  if (inputs.size() > 0) \{
    compiler.InitializeSourceManager(inputs[0].second);
    clang::ParseAST(
      pp,
      &compiler.getASTConsumer(),
      compiler.getASTContext()
    );
  \}
  return 0;
\}
  \end{semiverbatim}

  \psframe[linewidth=0pt,fillcolor=gray!10!white,fillstyle=solid,framearc=.2](3,6)(8,4)
  \rput[lb](3.1,5.5){・{\tt \_\_builtin\_strcmp}}
  \rput[lb](3.1,5.1){・{\tt \_\_builtin\_sin}}
  \rput[lb](3.1,4.7){・{\tt \_\_builtin\_cos}}
  \rput[lb](3.1,4.3){・{\tt \_\_builtin\_sqrt} ...}
\end{frame}

\begin{frame}[containsverbatim]
  \frametitle{clang::ParseAST}
  \begin{semiverbatim}
  (続)
  \color{gray}auto& pp = compiler.getPreprocessor();
  pp.getBuiltinInfo().InitializeBuiltins(
      pp.getIdentifierTable(), pp.getLangOptions());

  auto& inputs = compiler.getFrontendOpts().Inputs;
  if (inputs.size() > 0) \{
    \color{black}compiler.InitializeSourceManager(inputs[0].second);
    clang::ParseAST(
      pp,
      &compiler.getASTConsumer(),
      compiler.getASTContext()
    );
  \color{gray}\}
  return 0;
\}
  \end{semiverbatim}

  \psframe[linewidth=0pt,fillcolor=gray!10!white,fillstyle=solid,framearc=.2](1.3,2)(10,1)
  \rput[lb](1.5,1.3){読込$\rightarrow$PP$\rightarrow$解析(AST)$\rightarrow$AST Consumer}
\end{frame}

\begin{frame}[containsverbatim]
  \frametitle{consumer}

  \begin{semiverbatim}struct consumer : clang::ASTConsumer \{
  explicit
  consumer(std::string const& target);
  virtual
  void Initialize(clang::ASTContext& ctx) /*override*/;
  virtual
  void HandleTopLevelDecl(clang::DeclGroupRef decls) /*override*/;
private:
  void handle_functiondecl(clang::FunctionDecl const* fd) const;

  std::string target_;
  clang::ASTContext* ctx_;
\};\end{semiverbatim}

  {\scriptsize
    {\tt boost::optional<T\&>}のオーバヘッドが小さければなぁ
  }
\end{frame}

\begin{frame}[containsverbatim]
  \frametitle{consumer::consumer, Initialize}

  \begin{semiverbatim}\color{gray}struct consumer : clang::ASTConsumer \{
  \color{black}explicit
  consumer(std::string const& target)
  : target_(target), ctx_(nullptr) \{
  \}
  virtual
  void Initialize(clang::ASTContext& ctx) /*override*/ \{
    ctx_ = &ctx;
  \}
  \color{gray}virtual
  void HandleTopLevelDecl(clang::DeclGroupRef decls) /*override*/;
private:
  std::string target_;
  clang::ASTContext* ctx_;
\};\end{semiverbatim}
\end{frame}

\begin{frame}[containsverbatim]
  \frametitle{consumer::HandleTopLevelDecl}

  \begin{semiverbatim}\color{gray}struct consumer : clang::ASTConsumer \{
  explicit
  consumer(std::string const& target);
  virtual
  void Initialize(clang::ASTContext& ctx) /*override*/;
  \color{black}virtual
  void HandleTopLevelDecl(clang::DeclGroupRef decls) \{
    for (auto& decl : decls)
      if (auto const* fd =
          llvm::dyn_cast<clang::FunctionDecl>(decl))
        handle_functiondecl(fd);
  \}
\color{gray}private:
  void handle_functiondecl(clang::FunctionDecl const* fd) const;

  std::string target_;
  clang::ASTContext* ctx_;
\};\end{semiverbatim}
\end{frame}

\begin{frame}
  \begin{alertblock}{宣言, clang/AST/Decl*.h}
    \begin{itemize}
      \setlength{\itemsep}{0.5zh}
      \item Decl
      \item EnumDecl
      \item FunctionDecl
      \item TempladeDecl
      \item TypeDecl ...
    \end{itemize}
  \end{alertblock}

  \begin{alertblock}{文, clang/AST/Stmt*.h}
    \begin{itemize}
      \setlength{\itemsep}{0.5zh}
      \item Stmt
      \item IfStmt
      \item DoStmt
      \item ForStmt
      \item ReturnStmt ...
    \end{itemize}
  \end{alertblock}
\end{frame}

\begin{frame}
  \begin{alertblock}{式, clang/AST/Expr*.h}
    \begin{itemize}
      \setlength{\itemsep}{0.5zh}
      \item Expr
      \item IntegerLiteralExpr
      \item BinaryOperator
      \item CallExpr
      \item InitListExpr ...
    \end{itemize}
  \end{alertblock}
\end{frame}

\begin{frame}[containsverbatim]
  \frametitle{consumer::handle\_functiondecl}

  \begin{semiverbatim}\color{gray}struct consumer : clang::ASTConsumer \{
  explicit
  consumer(std::string const& target);
  virtual
  void Initialize(clang::ASTContext& ctx) /*override*/;
  virtual
  void HandleTopLevelDecl(clang::DeclGroupRef decls) /*override*/;
private:
  \color{black}void handle_functiondecl(clang::FunctionDecl const* fd) const \{
    if (fd->getDeclName().getAsString() == target_) \{
      auto const& range = fd->getSourceRange();
      range.getBegin().print(
        llvm::outs(), ctx_->getSourceManager());
      llvm::outs() << '¥n';
    \}
  \}
  \color{gray}std::string target_;
  clang::ASTContext* ctx_;
\};\end{semiverbatim}
\end{frame}

\begin{frame}[containsverbatim]
  \frametitle{Eratta}

  入力ソースにエラーがあると… 落ちます.

  \begin{alertblock}{error}
    \begin{semiverbatim}in.cpp:8:3: error: use of undeclared identifier 'a'
      Stack dump:
      0.	in.cpp:8:4: current parser token ';'
      1.	in.cpp:6:10: parsing function body 'g'
      2.	in.cpp:6:10: in compound statement ('{}')
      zsh: segmentation fault  ./where in.cpp f\end{semiverbatim}
  \end{alertblock}
\end{frame}

\begin{frame}[containsverbatim]
  \frametitle{Eratta}

  名前空間を解釈できません.

  \begin{alertblock}{in2.cpp}
\begin{semiverbatim}namespace ns \{
  void f() \{\}
\}\end{semiverbatim}
  \end{alertblock}

  \begin{alertblock}{shell}
\begin{semiverbatim}\$ ./where in2.cpp f
\$ ./where in2.cpp ns::f\end{semiverbatim}
  \end{alertblock}

  \vspace{.5zh}

  $\rightarrow$ clang::NamespaceDecl を再帰的に処理しないといけない.
\end{frame}

\begin{frame}[containsverbatim]
  \frametitle{組込み AST Consumer}

  \begin{itemize}
    \item clang/Frontent/ASTConsumers.h
      \begin{itemize}
        \setlength{\itemsep}{1.5zh}
        \item CreateASTPrinter
        \item CreateASTDumper
        \item CreateASTDumperXML
        \item CreateASTViewer
        \item CreateDeclContextPrinter
      \end{itemize}
  \end{itemize}

  \vspace{.5zh}

  {\small 手元の環境だとCreateASTDumperXMLが動かなかった…}
\end{frame}

\begin{frame}[containsverbatim]
  \frametitle{clang::CreateASTDumper()}

  \begin{semiverbatim}
int f(int x, int y) (CompoundStmt 0x7f87e4917b00 
                     <in.cpp:2:21, line:4:1>
 (ReturnStmt 0x7f87e4917ae0 <line:3:3, col:14>
  (BinaryOperator 0x7f87e4917ab8 <col:10, col:14> 'int' '+'
   (ImplicitCastExpr 0x7f87e4917a88 <col:10>
    'int' <LValueToRValue>
    (DeclRefExpr 0x7f87e4917a38 <col:10> 'int'
     lvalue ParmVar 0x7f87e49178c0 'x' 'int'))
    (ImplicitCastExpr 0x7f87e4917aa0 <col:14>
     'int' <LValueToRValue>
     (DeclRefExpr 0x7f87e4917a60 <col:14> 'int'
      lvalue ParmVar 0x7f87e4917920 'y' 'int')))))\end{semiverbatim}
\end{frame}

\begin{frame}
  \begin{center}
    {\LARGE\it Any Questions?}
  \end{center}
\end{frame}

%%\begin{frame}
%%  \psgrid(0,0)(12,9)
%%\end{frame}

\end{document}
